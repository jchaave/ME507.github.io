Overall, our system worked well! We could get the two sensors to work and the motors to work with them. Both worked together to keep the vehicle from colliding into objects even when the motors were working at top speed. We were also able to demonstrate this in the link provided below\+: ~\newline
 ~\newline
 Video Demonstration\+: \href{https://www.youtube.com/watch?v=Pw0nNdHIDUE&ab_channel=MatthewTagupa}{\texttt{ https\+://www.\+youtube.\+com/watch?v=\+Pw0n\+Nd\+H\+I\+D\+U\+E\&ab\+\_\+channel=\+Matthew\+Tagupa}}

~\newline
 Although the video above makes our project seem successful, we were not entirely satisfied with our results. In the code, there are snippets that indicate that the car is supposed to turn when it detects an object with both sensors. However, due to a physical design issues, the car cannot turn on its own. That is why in our video, we need to turn the car manually. The issue is that the two front tires spin together on one shaft. The two wheels cannot rotate to change their direction so the friction underneath the tires keeps the vehicle from rotating. When we were first testing, we were not able to see this issue because the vehicle was supported by a test stand. If we were to return to the design phase for the vehicle, we though it would be more beneficial to remove one of the front wheels and move the remaining front wheel to rotate freely. This way, the two rear wheels can control steering and the third wheel would be able to rotate without the issue of friction after the third wheel aligns in the direction that the vehicle is rotating. ~\newline
 ~\newline
 Another issue that occurred a few times involved soldering issues. We first wanted to get the motor driver chips soldered on since we knew those would be the most difficult to install. Our first attempt at soldering our board, a team member had soldered some of the leads on a transistor together, so that transistor ended up fried and we were left to try again. Due to the current pandemic, only two of our members were able to meet each other and those members needed to solder their own boards together. Our attempt was successful, but there was a lot of struggle involved. We needed to use a heat gun to solder the motor driver chips on since we were not able to solder the pins on the motor driver without crossing solder over pins that were not supposed to be touching. Our first time was unsuccessful, and we ended up soldering all the motor driver leads together. We struggled to get that solder off, so we tried again with two new motor driver chips. From this, experience we learned a great multitude of skills about soldering. When soldering the chips on, it is easier to apply solder to the locations where the chip and its pins need to go prior to soldering the chips on. We also learned that when using a heat gun, we do not recommend using a fan nearby even if there is lead solder in use. The fan seems to not allow the heat gun to supply a sufficient and constant amount of heat to the board when necessary. Also, it is important that when using the heat gun to lightly move chips in small increments because if you move it too much, the chip may slide off the solder and end up spreading the solder across the whole circuit board. If this occurs, we also recommend using solder removal, such as Chip\+Quik, along with a solder sucker to remove all the excess solder and then use rubbing alcohol to clean off the surface of the board. ~\newline
 ~\newline
 Although we had so many issues, we are excited about what we were able to complete and our team is ambitious to continue to work on mechatronics projects like this and more. Even with the current pandemic which made meeting each other difficult and made completing the project worse, we all enjoyed our collaboration and company as we worked toward our goal. We all are looking forward to all the projects that we want to complete now that we have the knowledge and skill to create our own mechanical control systems using C++ and Eagle. ~\newline
 ~\newline
  ~\newline
